\usepackage[a4paper,margin=2cm]{geometry}
\usepackage{xcolor}

%- Font
\usepackage[utf8]{inputenc}
% \usepackage{mathpazo}
% \usepackage{mathptmx}
% \usepackage[T1]{fontenc}



%- Babel
\usepackage[english]{babel}



%- Bibliogrophy
\usepackage[square,numbers]{natbib}
\bibliographystyle{unsrtnat}



%- Hyperref

\definecolor{myCiteColor}{RGB}{68, 118, 102}
\usepackage[colorlinks=true,citecolor=myCiteColor]{hyperref}
\pdfstringdefDisableCommands{\def\varepsilon{\textepsilon}}
\usepackage{bookmark}% faster updated bookmarks



%- Table
\usepackage{booktabs}



%- tikz
\usepackage{tikz}
\usetikzlibrary{arrows.meta,shapes,fadings,chains,calc}
\tikzset{%
    base/.style={rectangle, minimum width=3.5cm, minimum height=0.8cm,fill=gray!10},
}
% For use to mark a point for example in an equation
\newcommand{\tikzmark}[1]{\tikz[baseline,remember picture] \coordinate (#1) {};}




%- Math stuff
\usepackage{amsmath, amsfonts, mathtools, amsthm, amssymb}
\usepackage{resizegather}  % Automatically resize overly large equations
\usepackage{nicefrac}

% \newtheorem{theorem}{Theorem}[section]
% \newtheorem{lemma}[theorem]{Lemma}
\newcommand{\ra}[1]{\renewcommand{\arraystretch}{#1}}
\newcommand{\degree}{\ensuremath{^\circ\,}}
\newcommand{\overbar}[1]{\mkern 1.5mu\overline{\mkern-1.5mu#1\mkern-1.5mu}\mkern 1.5mu}
\newcommand{\f}{\frac}
\newcommand{\nf}{\nicefrac}
\newcommand{\tr}[1]{\mathrm{Tr}\left({#1}\right)}
\renewcommand{\d}{\mathop{}\!\mathrm{d}} % total derivative
\newcommand{\deltaop}{\mathop{}\!\delta} % total derivative
\newcommand{\D}{\mathop{}\!\mathrm{D}} % total derivative
\newcommand{\p}{\partial}
\newcommand{\rmm}{\mathrm}
\newcommand{\mr}{\mathrm}
\newcommand{\ddt}[1]{\frac{\partial #1}{\partial t}}
\newcommand{\ddx}[1]{\frac{\partial #1}{\partial x}}
\newcommand{\ddy}[1]{\frac{\partial #1}{\partial y}}
\newcommand{\ddz}[1]{\frac{\partial #1}{\partial z}}
\newcommand{\ddtt}[1]{\frac{\partial^2 #1}{\partial t^2}}
\newcommand{\ddxx}[1]{\frac{\partial^2 #1}{\partial x^2}}
\newcommand{\ddxy}[1]{\frac{\partial^2 #1}{\partial x \partial y}}
\newcommand{\ddxz}[1]{\frac{\partial^2 #1}{\partial x \partial z}}
\newcommand{\ddyx}[1]{\frac{\partial^2 #1}{\partial y \partial x}}
\newcommand{\ddyy}[1]{\frac{\partial^2 #1}{\partial y2}}
\newcommand{\ddyz}[1]{\frac{\partial^2 #1}{\partial y \partial z}}
\newcommand{\ddzx}[1]{\frac{\partial^2 #1}{\partial z \partial x}}
\newcommand{\ddzy}[1]{\frac{\partial^2 #1}{\partial z \partial y}}
\newcommand{\ddzz}[1]{\frac{\partial^2 #1}{\partial z^2}}
%
\newcommand{\ddtout}[1]{\frac{\partial }{\partial t}{#1}}
\newcommand{\ddxout}[1]{\frac{\partial }{\partial x}{#1}}
\newcommand{\ddyout}[1]{\frac{\partial }{\partial y}{#1}}
\newcommand{\ddzout}[1]{\frac{\partial }{\partial z}{#1}}
\newcommand{\ddttout}[1]{\frac{\partial^2 }{\partial t^2}{#1}}
\newcommand{\ddxxout}[1]{\frac{\partial^2 }{\partial x^2}{#1}}
\newcommand{\ddxyout}[1]{\frac{\partial^2 }{\partial x \partial y}{#1}}
\newcommand{\ddxzout}[1]{\frac{\partial^2 }{\partial x \partial z}{#1}}
\newcommand{\ddyxout}[1]{\frac{\partial^2 }{\partial y \partial x}{#1}}
\newcommand{\ddyyout}[1]{\frac{\partial^2 }{\partial y2}{#1}}
\newcommand{\ddyzout}[1]{\frac{\partial^2 }{\partial y \partial z}{#1}}
\newcommand{\ddzxout}[1]{\frac{\partial^2 }{\partial z \partial x}{#1}}
\newcommand{\ddzyout}[1]{\frac{\partial^2 }{\partial z \partial y}{#1}}
\newcommand{\ddzzout}[1]{\frac{\partial^2 }{\partial z^2}{#1}}
%
% Total derivative
\newcommand{\DDt}[1]{\frac{\partial #1}{\p t} + u\frac{\p #1}{\p x} + v\frac{\p #1}{\p y} + w\frac{\p #1}{\p z}}
% Total derivative vector notation
\newcommand{\DDtv}[1]{\frac{\partial #1}{\p t} + \left( \U \cdot \grad \right) #1}
% Total derivative itself
\newcommand{\DDT}[1]{\frac{\D #1}{\D t}}

% Gradient
\newcommand{\grad}{\mathop{}\!\nabla} % total derivative
% % Gradient in case subscript/superscript is needed
% \NewDocumentCommand{\grad}{e{_^}}{%
%   \mathop{}\!% \mathop for good spacing before \nabla
%   \nabla
%   \IfValueT{#1}{_{\!#1}}% tuck in the subscript
%   \IfValueT{#2}{^{#2}}% possible superscript
% }

\newcommand{\mb}{\mathbf} % bold symbols in math
\newcommand{\mbg}[1]{\boldsymbol{\mathbf{#1}}} % bold for Greek symbols
\usepackage{bm} % in case we want to have bold symbols (vectors) in italic
% \bm{\alpha} \bm{u}

\newcommand{\U}{\mathbf{U}}

\newcommand{\eqnref}[1]{Eq.~\ref{#1}}

% Put x \to \infty below \lim
\let\svlim\lim\def\lim{\svlim\limits}
%Make implies and impliedby shorter
\let\implies\Rightarrow
\let\impliedby\Leftarrow
\let\iff\Leftrightarrow

% Swap epsilon and varepsilon
\let\epsilonOld\epsilon
\let\epsilon\varepsilon   %<------- epsilon == varepsilon
\let\varepsilon\epsilonOld

% Add \contra symbol to denote contradiction
\usepackage{stmaryrd} % for \lightning
\newcommand\contra{\scalebox{1.5}{$\lightning$}}

% Fancy script capitals
\usepackage{mathrsfs}
\usepackage{cancel}
% Bold math
\usepackage{bm}
% Some shortcuts
\newcommand\N{\ensuremath{\mathbb{N}}}
\newcommand\R{\ensuremath{\mathbb{R}}}
\newcommand\Z{\ensuremath{\mathbb{Z}}}
\renewcommand\O{\ensuremath{\emptyset}}
\newcommand\Q{\ensuremath{\mathbb{Q}}}
\newcommand\C{\ensuremath{\mathbb{C}}}

% Circle around a math term
% \mathcircled{math}
\makeatletter
\newcommand\mathcircled[1]{%
  \mathpalette\@mathcircled{#1}%
}
\newcommand\@mathcircled[2]{%
  \tikz[baseline=(math.base)] \node[draw,circle,inner sep=1pt] (math) {$\m@th#1#2$};%
}
\makeatother

% Theorems
\newtheorem*{remark}{Remark}






%- clever reference (must be after hyperref and amsmath)
\usepackage{cleveref}
\Crefname{equation}{Eq.}{Eqs.}
\Crefname{figure}{Fig.}{Figs.}
% \Crefname{tabular}{Tab.}{Tabs.}




%- Figures
\usepackage{graphicx}
\graphicspath{{./figures}}
\usepackage[%
% reduce the empty space above the caption
% https://tex.stackexchange.com/a/45996/220469
%\setlength{\abovecaptionskip}{-3pt plus 0pt minus 0pt}
% or
%
skip=2pt,%
% also make the caption label bold
labelfont=bf]{caption}

\usepackage{adjustbox}

% Figure support for Inkscape
\usepackage{import}
\usepackage{xifthen}
\usepackage{pdfpages}
\usepackage{transparent}
\newcommand{\incfig}[1]{%
    \def\svgwidth{\columnwidth}
    \import{./figures/}{#1.pdf_tex}
}




%- Deal with the annoying overfull hbox non-sense
\tolerance=1000
\hbadness=10000
\emergencystretch=\maxdimen
\hyphenpenalty=1000
\hfuzz=0.1pt




%- Code listings
\definecolor{LightGrey}{HTML}{fafafa}
\definecolor{codegreen}{rgb}{0,0.6,0}
\definecolor{codegray}{rgb}{0.5,0.5,0.5}
\definecolor{codepurple}{rgb}{0.58,0,0.82}
\definecolor{backcolour}{rgb}{0.95,0.95,0.92}

% % \usepackage{listings}
% % \lstdefinestyle{mystyle}{
% %     backgroundcolor=\color{backcolour},
% %     commentstyle=\color{codegreen},
% %     keywordstyle=\color{magenta},
% %     numberstyle=\tiny\color{codegray},
% %     stringstyle=\color{codepurple},
% %     basicstyle=\ttfamily\footnotesize,
% %     breakatwhitespace=false,
% %     breaklines=true,
% %     captionpos=b,
% %     keepspaces=true,
% %     numbers=left,
% %     numbersep=5pt,
% %     showspaces=false,
% %     showstringspaces=false,
% %     showtabs=false,
% %     tabsize=2,
% %     escapeinside={/@}{@/)}, % escape characters to write latex code in the listing (e.g. highlighting) (  (*@  \textcolor{blue}{code}  @*) )
% %     % columns=fullflexible,  % fixes issues related to spacing of the tex code inside lstlistings
% %     moredelim=[is][\underbar]{_}{_},
% % }
% % \lstset{ style=mystyle }
% \usepackage{color} % for highlight purpose
% \usepackage[cache=false,outputdir=build]{minted}
% \definecolor{LightBrown}{rgb}{0.95,0.95,0.92}
% \definecolor{LightGrey}{HTML}{fafafa}
% \definecolor{MidGrey}{HTML}{8a8a8a}
% % \setminted[<language>]{<key=value,...>}
% \setminted{
% fontsize=\footnotesize,
% bgcolor=LightBrown,
% }
% \renewcommand{\theFancyVerbLine}{\textcolor{MidGrey}{\arabic{FancyVerbLine}}}
% % reduce the space between caption and listing
% \AtEndEnvironment{listing}{\vspace{-20pt}}





%- Units
\usepackage{siunitx}




%- Paths and urls
% xurl: a seemingly better version of url, which solves the problem long urls/paths not breaking.
\usepackage{xurl}




%- Colored boxed for amsmath
% https://tex.stackexchange.com/a/326380/220469
% Definition of \boxed in amsmath.sty:
% \newcommand{\boxed}[1]{\fbox{\m@th$\displaystyle#1$}}
\definecolor{LightBlue}{HTML}{e6feff}
%stroke-fill-hl
\newcommand{\sfhl}{\fcolorbox{LightBlue}{LightBlue}}
%stroke-fill-hl-mono (mono version)
\newcommand{\sfhlmono}[1]{\fcolorbox{LightBlue}{LightBlue}{\texttt{#1}}}

% Syntax: \colorboxed[<color model>]{<color specification>}{<math formula>}
\newcommand*{\colorboxed}{}
\def\colorboxed#1#{%
  \colorboxedAux{#1}%
}
\newcommand*{\colorboxedAux}[3]{%
  % #1: optional argument for color model
  % #2: color specification
  % #3: formula
  \begingroup
    \colorlet{cb@saved}{.}%
    \color#1{#2}%
    \boxed{%
      \color{cb@saved}%
      #3%
    }%
  \endgroup
}




%- chapter settings
% Use "Lecture" as chaptername
\makeatletter
\renewcommand{\@chapapp}{Lecture}
\makeatother




%- Some nice helpers
\usepackage{xspace}
\newcommand{\keModel}{\(k\text{-}\epsilon\)\xspace}
\newcommand{\mono}{\texttt}



%- There's a chpater-0 and for that we need to start numbering equations/figures from chapter 0
% https://tex.stackexchange.com/a/167032/220469
\renewcommand{\theequation}{\thechapter.\arabic{equation}}
\renewcommand{\thefigure}{\thechapter.\arabic{figure}}




\usepackage{stmaryrd}
\SetSymbolFont{stmry}{bold}{U}{stmry}{m}{n}
