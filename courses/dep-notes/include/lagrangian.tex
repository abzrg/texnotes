\chapter{Lagrangian}

\section{\emph{Lagrangian} and \emph{Eulerian}}

\begin{itemize}
  \item \B{Lagrangian} specification of the flow field In classical field theory, the Lagrangian specification of the field is a way of looking at fluid motion where the observer follows an individual fluid parcel as it moves through space and time~\cite{Xu_2016,Batchelor_2000}.
  \item \B{Eulerian} specification of the flow field The Eulerian specification of the flow field is a way of looking at fluid motion that focuses on specific locations in the space through which the fluid flows as time passes~\cite{Xu_2016}.
\end{itemize}


\section{OpenFOAM Lagrangian Library}



Three concept:
\begin{itemize}
  \item \B{Cloud} is a collection of parcels with different physical properties.
  \item \B{Parcel} is a collection of particles with the same physical properties.
  \item \B{Particle} is responsible for tracking position in the mesh.
\end{itemize}
Actual physics in the parcel~\cite{Zaripov_2018}. Relevant files are located at:
\T{\$FOAM\_SRC\slash lagrangian\slash intermediate\slash parcels\slash Templates\slash KinematicParcel\slash}

% \begin{lstlisting}[language=C++]
% \end{lstlisting}

In all the LPT~\footnote{\B{L}agrangian \B{P}article \B{T}racking} libraries of OpenFOAM, the
dispersed phase are described by two major structure: \emph{Cloud} and \emph{Parcel}. Cloud is a
list of parcels plus some general properties for every parcel (note the Cloud here is not relevant
to the template class \texttt{Cloud}~\cite{Chen_2013}.

In the simplest case, \texttt{solidParticleCloud} is a list of \T{solidParticle}. Another
example being \texttt{KinematicCloud} with \texttt{KinematicParcel}.~\cite{Chen_2013}

Two additional structure worth notice: \T{particle} and \T{Cloud}. They're both defined in the
\T{lagrangian\slash basic} library and being applied in all the LPT libraries. \T{Particle} is a class
that contains position (and it's tracked in any library); \T{Cloud} is a list of
\T{parcel}.~\cite{Chen_2013}


