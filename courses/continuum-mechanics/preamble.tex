%- Font
\usepackage[utf8]{inputenc}
% \usepackage[T1]{fontenc}
% \usepackage{cmbright}
% \usepackage{mathptmx}
% \usepackage{kpfonts}
% \usepackage{cmupint}  % upright integrals for CMU

% Always load it
\usepackage[activate={true,nocompatibility},final,tracking=true,kerning=true,spacing=false,factor=1100,stretch=10,shrink=10]{microtype}
\AtBeginEnvironment{verbatim}{\microtypesetup{activate=false}} % To avoid problems with verbatim-areas



%- Title
\title{Continuum Mechanics}
\author{Ali Bozorgzadeh}
\date{}



%- Pile of Packages
\usepackage[a4paper,margin=2cm]{geometry}
\usepackage{xcolor}
\usepackage[english]{babel}
\usepackage{enumitem}  % To Tweak lists
\usepackage{caption}  % Enhance caption
\usepackage{float} % Control floating environment (provides 'H' option for HERE!)
% \definecolor{myCiteColor}{HTML}{000080}
\definecolor{myURLColor}{HTML}{0000a0}
\usepackage[colorlinks=false,hidelinks]{hyperref}
% \pdfstringdefDisableCommands{\def\varepsilon{\textepsilon}}
\usepackage{bookmark}  % Faster updated bookmarks
\usepackage{url}  % Have URLs and file paths that can break
\usepackage{booktabs}  % Table
\usepackage{multicol}
\usepackage{mdframed}  % Brakeable framed and colored boxes
% \usepackage{lipsum}


%- Figures
\usepackage{graphicx}
\graphicspath{{./figures}}
\usepackage[%
	% reduce the empty space above the caption
	% https://tex.stackexchange.com/a/45996/220469
	%\setlength{\abovecaptionskip}{-3pt plus 0pt minus 0pt}
	% or
	%
	skip=2pt,%
	% also make the caption label bold
	labelfont=bf]{caption}

% Figure support for Inkscape
\usepackage{import}
\usepackage{xifthen}
\usepackage{pdfpages}
\usepackage{transparent}
\newcommand{\incfig}[1]{%
	\def\svgwidth{\columnwidth}
	\import{./figures/}{#1.pdf_tex}
}




%- Math stuff
\usepackage{amsfonts,amssymb,amsthm,mathtools}  % don't directly load amsmath
\usepackage{textcomp}  % More symbols?
\usepackage{resizegather}  % Automatically resize overly large equations
\usepackage{nicefrac}

% Useful commands
\newcommand{\ra}[1]{\renewcommand{\arraystretch}{#1}}
\newcommand{\degree}{\ensuremath{^\circ\,}}
\newcommand{\overbar}[1]{\mkern 1.5mu\overline{\mkern-1.5mu#1\mkern-1.5mu}\mkern 1.5mu}
\newcommand{\f}{\frac}
\newcommand{\nf}{\nicefrac}
\newcommand{\tr}[1]{\mathrm{Tr}\left({#1}\right)}
\renewcommand{\d}{\mathop{}\!\mathrm{d}} % total derivative
\newcommand{\D}{\mathop{}\!\mathrm{D}} % total derivative
\newcommand{\deltaop}{\mathop{}\!\delta} % delta for small length in each direction (\deltaop x)
\newcommand{\p}{\partial}
\newcommand{\rmm}{\mathrm}
\newcommand{\mr}{\mathrm}

% Functions that return real and imaginary part of a complex number
\usepackage{mleftright} % For better versions of \left and \right (\mleft and \mright)
\newcommand{\Real}[2][]{\ensuremath{\mathop{}\!\Re^{#1}\mleft(#2\mright)}}
\newcommand{\Imag}[2][]{\ensuremath{\mathop{}\!\Im^{#1}\mleft(#2\mright)}}

% Some shortcuts
\newcommand{\U}{\mathbf{U}} % Velocity vector
\newcommand{\E}{\mathbf{E}} % Electric field vector
\newcommand{\F}{\mathbf{F}} % Force vector
\newcommand{\gammadot}{\dot{\gamma}} % shear rate

% Differential operators
\newcommand{\ddt}[1]{\frac{\partial #1}{\partial t}}
\newcommand{\ddx}[1]{\frac{\partial #1}{\partial x}}
\newcommand{\ddy}[1]{\frac{\partial #1}{\partial y}}
\newcommand{\ddz}[1]{\frac{\partial #1}{\partial z}}
\newcommand{\ddtt}[1]{\frac{\partial^2 #1}{\partial t^2}}
\newcommand{\ddxx}[1]{\frac{\partial^2 #1}{\partial x^2}}
\newcommand{\ddxy}[1]{\frac{\partial^2 #1}{\partial x \partial y}}
\newcommand{\ddxz}[1]{\frac{\partial^2 #1}{\partial x \partial z}}
\newcommand{\ddyx}[1]{\frac{\partial^2 #1}{\partial y \partial x}}
\newcommand{\ddyy}[1]{\frac{\partial^2 #1}{\partial y2}}
\newcommand{\ddyz}[1]{\frac{\partial^2 #1}{\partial y \partial z}}
\newcommand{\ddzx}[1]{\frac{\partial^2 #1}{\partial z \partial x}}
\newcommand{\ddzy}[1]{\frac{\partial^2 #1}{\partial z \partial y}}
\newcommand{\ddzz}[1]{\frac{\partial^2 #1}{\partial z^2}}
%
\newcommand{\ddtout}[1]{\frac{\partial }{\partial t}{#1}}
\newcommand{\ddxout}[1]{\frac{\partial }{\partial x}{#1}}
\newcommand{\ddyout}[1]{\frac{\partial }{\partial y}{#1}}
\newcommand{\ddzout}[1]{\frac{\partial }{\partial z}{#1}}
\newcommand{\ddttout}[1]{\frac{\partial^2 }{\partial t^2}{#1}}
\newcommand{\ddxxout}[1]{\frac{\partial^2 }{\partial x^2}{#1}}
\newcommand{\ddxyout}[1]{\frac{\partial^2 }{\partial x \partial y}{#1}}
\newcommand{\ddxzout}[1]{\frac{\partial^2 }{\partial x \partial z}{#1}}
\newcommand{\ddyxout}[1]{\frac{\partial^2 }{\partial y \partial x}{#1}}
\newcommand{\ddyyout}[1]{\frac{\partial^2 }{\partial y2}{#1}}
\newcommand{\ddyzout}[1]{\frac{\partial^2 }{\partial y \partial z}{#1}}
\newcommand{\ddzxout}[1]{\frac{\partial^2 }{\partial z \partial x}{#1}}
\newcommand{\ddzyout}[1]{\frac{\partial^2 }{\partial z \partial y}{#1}}
\newcommand{\ddzzout}[1]{\frac{\partial^2 }{\partial z^2}{#1}}
%
% Total derivative
\newcommand{\DDt}[1]{\frac{\partial #1}{\p t} + u\frac{\p #1}{\p x} + v\frac{\p #1}{\p y} + w\frac{\p #1}{\p z}}

% Total derivative vector notation
\newcommand{\DDtv}[1]{\frac{\partial #1}{\p t} + \left( \U \cdot \grad \right) #1}

% Total derivative itself
\newcommand{\DDT}[1]{\frac{\D #1}{\D t}}

% Gradient
\newcommand{\grad}{\mathop{}\!\nabla} % total derivative
% % Gradient in case subscript/superscript is needed
% \NewDocumentCommand{\grad}{e{_^}}{%
%   \mathop{}\!% \mathop for good spacing before \nabla
%   \nabla
%   \IfValueT{#1}{_{\!#1}}% tuck in the subscript
%   \IfValueT{#2}{^{#2}}% possible superscript
% }

% Bold math symbols
\newcommand{\mb}{\mathbf}
\newcommand{\mbg}[1]{\boldsymbol{\mathbf{#1}}} % bold for Greek symbols
\usepackage{bm} % in case we want to have bold symbols (vectors) in italic: \bm{\alpha} \bm{u}

% Put x \to \infty below \lim
\let\svlim\lim\def\lim{\svlim\limits}

%Make implies and impliedby shorter
\let\implies\Rightarrow
\let\impliedby\Leftarrow
\let\iff\Leftrightarrow

% Swap epsilon and varepsilon
\let\epsilonOld\epsilon
\let\epsilon\varepsilon   %<------- epsilon == varepsilon
\let\varepsilon\epsilonOld

% Add \contra symbol to denote contradiction
\usepackage{stmaryrd} % for \lightning
\SetSymbolFont{stmry}{bold}{U}{stmry}{m}{n}  % SILENCE! https://tex.stackexchange.com/a/106719/220469
\newcommand\contra{\scalebox{1.5}{$\lightning$}}

% Fancy script capitals
\usepackage{mathrsfs}

% Cancel terms
\usepackage{cancel}

% Theorem environments
\newtheorem{theorem}{Theorem}[section]
\newtheorem{lemma}[theorem]{Lemma}
\newtheorem*{remark}{Remark}


% Small underbrace
\makeatletter
\def\smallunderbrace#1{\mathop{\vtop{\m@th\ialign{##\crcr
				$\hfil\displaystyle{#1}\hfil$\crcr
				\noalign{\kern3\p@\nointerlineskip}%
				\tiny\upbracefill\crcr\noalign{\kern3\p@}}}}\limits}
\makeatother

% Absolute value and norm function
\DeclarePairedDelimiter\abs{\lvert}{\rvert}%
\DeclarePairedDelimiter\norm{\lVert}{\rVert}%
% Swap the definition of \abs* and \norm*, so that \abs
% and \norm resizes the size of the brackets, and the 
% starred version does not.
\makeatletter
\let\oldabs\abs
\def\abs{\@ifstar{\oldabs}{\oldabs*}}
%
\let\oldnorm\norm
\def\norm{\@ifstar{\oldnorm}{\oldnorm*}}
\makeatother





%- Bibliography
\usepackage[square,numbers]{natbib}
\bibliographystyle{unsrtnat}






%% Clever reference (must be after hyperref and amsmath)
\usepackage{cleveref}
\Crefname{equation}{Eq.}{Eqs.}
\Crefname{figure}{Fig.}{Figs.}
% \Crefname{tabular}{Tab.}{Tabs.}




%% Deal with the annoying overfull hbox non-sense
\tolerance=1000
\hbadness=10000
\emergencystretch=\maxdimen
\hyphenpenalty=1000
\hfuzz=0.1pt




%% Units
\usepackage{siunitx}
\DeclareSIUnit\sm{\mathrm{S}}  % siemens





% %% Nomenclature (needs special .latexmkrc)
% \usepackage[intoc]{nomencl}
% \makenomenclature
% \renewcommand{\nomname}{List of Symbols}  % Default title
% % \renewcommand{\nompreamble}{The next list describes several symbols that will be later used within the body of the document}  % Text in between the title and the list symbols
%
% % This will add the subgroups
% \usepackage{etoolbox}
% \renewcommand\nomgroup[1]{%
% 	\item[\bfseries
% 	            \ifstrequal{#1}{A}{Physics Constants}{%
% 		            \ifstrequal{#1}{B}{Number Sets}{%
% 			            \ifstrequal{#1}{C}{Other Symbols}{}}}%
% 	      ]}
%
% % This will add the units
% \newcommand{\nomunit}[1]{%
% 	\renewcommand{\nomentryend}{\hspace*{\fill}#1}}
%
% % .latexmkrc
% %
% % @cus_dep_list = (@cus_dep_list, "nlo nls 0 makenomenclature");
% % sub makenomenclature {
% %    system("makeindex $_[0].nlo -s nomencl.ist -o $_[0].nls"); }
% % @generated_exts = (@generated_exts, 'nlo');
% %





% %% Code listings
% \usepackage{listings}
% \usepackage{minted}
% \usepackage{upquote}  % Make single-quote inside verbatim text come out correctly







% % todonotes
% \usepackage{todonotes}
% \setuptodonotes{figwidth=0.5\textwidth,figcolor=orange!10,color=orange!40,inline,noline} % pass 'disable' to disable all todos
% \newcommand{\note}[1]{\todo[noline,inline,color=green!40,textcolor=black]{#1}}
% % \todo{default todo}
% % \todo[inline, color=red]{inline todo}
% % \todo[noline]{don't insert underline}
% % \missingfigure{caption}
% % \listoftodos
