\chapter{Theoretical Background}

\section{Preliminary}

The following is the Navier Stokes equations:
%
\begin{equation}
  \begin{gathered}
    \mu \nabla^2 u - \f{\p p}{\p x} + \rho g_x = \rho \left(\f{\D u}{\D t}\right), \\
    \mu \nabla^2 v - \f{\p p}{\p y} + \rho g_y = \rho \left(\f{\D v}{\D t}\right), \\
    \mu \nabla^2 w - \f{\p p}{\p z} + \rho g_z = \rho \left(\f{\D w}{\D t}\right), \\
    \nabla \cdot \mb{U} = 0,
  \end{gathered}
  \label{eq:navier-stokes}
\end{equation}

%
where \( \mb{U} = u \mb{i} + v \mb{j} + w \mb{k} \). The convective form of incompressible Navier-Stokes (by dividing for the density) is the following:
%
\begin{equation}
  \begin{gathered}
    \f{\p \mb{U}}{dt} + \left( \mb{U} \cdot \nabla \right) \mb{U} - \nu \nabla^2 \mb{U} = -\nabla w + \mb{g}, \\
    \f{1}{\rho_0} \nabla p = \nabla \left(\f{p}{\rho_0)} \right) \equiv \nabla w,
  \end{gathered}
  \label{eq:navier-stokes-compact}
\end{equation}
%
where \(w\) is the specific weight (with the sense of per unit mass) thermodynamic work, the internal source term.
The first two terms of the above equation can be replaced by the \emph{total derivative} (with \(\phi = \mb{U}\)):
%
\begin{equation}
\begin{aligned}
  \f{\D\phi}{\D t} &= \f{\p \phi}{\p t} + u \f{\p\phi}{\p x} + v \f{\p\phi}{\p y} + w \f{\p \phi}{\p z}, \\
               &= \f {\p\phi}{\p t} + \mb{U} \cdot \nabla\phi.
\end{aligned}
\label{eq:total-derivative}
\end{equation}


\section{Averaging Navier-Stokes}

The goal is to decompose the instantaneous velocity and pressure field into two components: a \emph{average field},
and a \emph{fluctuating field}.
%
\begin{equation}
  \begin{gathered}
  \mb{U} = \bar{\mb{U}} + \mb{U}^\prime, \\
  u = \bar{u} + u^\prime, \\
  v = \bar{v} + v^\prime, \\
  w = \bar{w} + w^\prime,
  \end{gathered}
  \label{eq:velocity-as-two-component}
\end{equation}
%
and
%
\begin{equation}
  p = \bar{p} + p^\prime,
  \label{eq:pressure-as-two-component}
\end{equation}
%
where the first term \(\bar{\square}\) is the averaged component and \(\square^\prime\) is the fluctuating component.
To do so, we need to do some \emph{time averaging} on the Navier-Stokes equation for velocity and pressure fields.
Gravity, however, does not usually fluctuate and thus we ignore it.
We start by substituting the Eq.~\ref{eq:velocity-as-two-component} into Eq.~\ref{eq:navier-stokes}.
We also expand the total derivative term on the right hand side as well.
%
\begin{equation}
    \mu \nabla^2 u - \f{\p p}{\p x} + \rho g_x
    = \rho \left(\f{\p u}{\p t} + u \f{\p u}{\p x} + v \f{\p u}{\p y} + w \f{\p u}{\p z} \right) \\
  \label{eq:NS-time-avg-x-comp}
\end{equation}
\begin{equation}
    \f{\p u}{\p x} + \f{\p v}{\p y} + \f{\p w}{\p z} = 0
  \label{eq:continuity-avg-x-comp}
\end{equation}
%
Then we substitute the components of the instantaneous fields.
%
\begin{multline}
    \mu \nabla^2 \left(\bar{u} + u^\prime\right) - \f{\p \left(\bar{p} + p^\prime\right)}{\p x} + \rho g_x \\
    = \rho \left(\f{\p \left(\bar{u} + u^\prime\right)}{\p t} + \left(\bar{u} + u^\prime\right) \f{\p \left(\bar{u} + u^\prime\right)}{\p x} + \left(\bar{v} + v^\prime\right) \f{\p \left(\bar{u} + u^\prime\right)}{\p y} + \left(\bar{w} + w^\prime\right) \f{\p \left(\bar{u} + u^\prime\right)}{\p z} \right),
  \label{eq:NS-time-avg-subst-comp}
\end{multline}
\begin{equation}
  \f{\p}{\p x}\left(\bar{u} + u^\prime\right)
  + \f{\p}{\p y}\left(\bar{v} + v^\prime\right)
  + \f{\p}{\p z}\left(\bar{w} + w^\prime\right) = 0.
\label{eq:continuity-avg-subst-comp}
\end{equation}



\subsection{Time averaging the continuity equation}

Expanding the Eq.~\ref{eq:continuity-avg-subst-comp},
%
\begin{equation}
  \f{\p \bar{u}}{\p x}
  + \f{\p \bar{v}}{\p y}
  + \f{\p \bar{w}}{\p z}
  + \f{\p u^\prime}{\p x}
  + \f{\p v^\prime}{\p y}
  + \f{\p w^\prime}{\p z} = 0.
  \label{eq:continuity-avg-subst-comp-expanded}
\end{equation}

In engineering, for the most part, we are not interested to those fluctuating components,
and want to capture the trend of the system.
We can get rid of some of these fluctuating term using the time averaging method.

\begin{figure}[h]
  \begin{center}
    \incfig{velocity-time-avg-and-fluctuating}
  \end{center}
  \caption{Veolcity (x-component) field vs time for a turbulent flow. Note the average component (black) and the fluctuating component (orange).}\label{fig:velocity-time-avg-and-fluctuating}
\end{figure}

The time average (\(\bar{f}\)) of a variable \(f\) is defined by
%
\begin{equation}
  \colorboxed{orange}{
    \bar{f} = \lim_{T\to\infty}{\f{1}{T} \int_{t_0}^{t_0 + T}{f \d t}}
  }
  \label{eq:time-average-formula}
\end{equation}
%
The averaging is done over a period, \(T\),
which must be short enough to properly capture the trend of the flow,
and yet long enough to get rid of the fluctuating behavior of the flow.
Note that time average of the fluctuating component is \(0\),
%
\begin{equation}
  \colorboxed{gray}{
    \overline{f^\prime} = \f{1}{T} \int_{t_0}^{t_0 + T}{f^\prime \d t} = 0,
  }
  \label{eq:time-average-fluctuating-comp}
\end{equation}
%
and time average of the time-averaged component is still an average function of time
%
\begin{equation}
  \colorboxed{gray}{
    \overline{\bar{f}} = \f{1}{T} \int_{t_0}^{t_0 + T}{\bar{f} \d t} = \bar{f}.
  }
  \label{eq:time-average-average-comp}
\end{equation}

There are other properties related to time averaging.
%
\begin{equation}
  \colorboxed{orange}{
    \begin{aligned}
      (a)&\quad \overline{f + g} = \bar{f} + \bar{g}, \\[5pt]
      (b)&\quad \overline{cf} = c \bar{f}\quad (c \text{ is a constant),} \\[5pt]
      (c)&\quad \overline{\bar{f}g} = \bar{f} \bar{g}, \\[5pt]
      (d)&\quad \overline{f^\prime g^\prime} \neq 0, \\[5pt]
      (e)&\quad \overline{fg} = \bar{f} \bar{g} + \overline{f^\prime g^\prime}, \\[5pt]
      (f)&\quad \overline{\f{\p f}{\p x}} = \f{\p \bar{f}}{\p x}, \\[5pt]
      (g)&\quad \overline{\int{f \d x}} = \int{\bar{f} \d x}.
    \end{aligned}
  }
  \label{eq:time-avgerage-props}
\end{equation}

After time-averaging the Eq.~\ref{eq:continuity-avg-subst-comp-expanded},
we'll cross out the fluctuating terms:
%
\begin{equation}
  \overline{\f{\p \bar{u}}{\p x}}
  + \overline{\f{\p \bar{v}}{\p y}}
  + \overline{\f{\p \bar{w}}{\p z}}
  + \bcancel{\overline{\f{\p u^\prime}{\p x}}}
  + \bcancel{\overline{\f{\p v^\prime}{\p y}}}
  + \bcancel{\overline{\f{\p w^\prime}{\p z}}} = 0.
  \label{eq:continuity-time-averaged-cancelled}
\end{equation}
%
Thus, after time-averaging the continuity equation looks like so
%
\begin{equation}
  \f{\p \bar{u}}{\p x}
+ \f{\p \bar{v}}{\p y}
+ \f{\p \bar{w}}{\p z} = 0.
  \label{eq:continuity-time-averaged-final-avg}
\end{equation}
%
As a result, we also have
\begin{equation}
  \f{\p {u^\prime}}{\p x}
+ \f{\p {v^\prime}}{\p y}
+ \f{\p {w^\prime}}{\p z} = 0.
  \label{eq:continuity-time-averaged-final-fluctuating}
\end{equation}


\subsection{Time averaging the x-momentum equation}

Starting by time-averaging the left hand side of the Eq.~\ref{eq:NS-time-avg-subst-comp}, we'll have
%
\begin{equation}
\begin{aligned}
    &\overline{\mu \nabla^2 \left(\bar{u} + u^\prime\right)} - \overline{\f{\p \left(\bar{p} + p^\prime\right)}{\p x}} + \overline{\rho g_x} \\
    \implies\quad &\mu \nabla^2 \left(\overline{\bar{u}} + \bcancel{\overline{u^\prime}}\right) - \f{\p \left(\overline{\bar{p}} + \bcancel{\overline{p^\prime}}\right)}{\p x} + \rho g_x \\
    \implies\quad &\mu \nabla^2 \left(\bar{u}\right) - \f{\p \left(\bar{p}\right)}{\p x} + \rho g_x.
\end{aligned}
  \label{eq:NS-time-averaged-lhs}
\end{equation}
%
Now, for the right hand side:
%
\begin{equation}
  \rho
  \left(
    \overline{\f{\p \left(\bar{u} + u^\prime\right)}{\p t}}
  + \overline{\left(\bar{u} + u^\prime\right) \f{\p \left(\bar{u} + u^\prime\right)}{\p x}}
  + \overline{\left(\bar{v} + v^\prime\right) \f{\p \left(\bar{u} + u^\prime\right)}{\p y}}
  + \overline{\left(\bar{w} + w^\prime\right) \f{\p \left(\bar{u} + u^\prime\right)}{\p z}}
  \right). \\
  \label{eq:NS-time-averaged-rhs}
\end{equation}
%
Except from the temporal derivative term, the other terms are non-linear.
We first start by time-averaging the time derivative term,
%
\begin{equation}
\begin{aligned}
  &\overline{\f{\p \left(\bar{u} + u^\prime\right)}{\p t}} \\
  \implies\quad &\f{\p \left(\overline{\bar{u}} + \bcancel{\overline{u^\prime}}\right)}{\p t} \\
  \implies\quad &\f{\p \left(\bar{u}\right)}{\p t}
\end{aligned}
  \label{eq:NS-time-averaged-rhs-ddt}
\end{equation}
%
For the non-linear terms we consider each one by one.
Starting from the first non-linear term, we'll have
%
\begin{equation}
  \begin{aligned}
  &\overline{\left(\bar{u} + u^\prime\right) \f{\p \left(\bar{u} + u^\prime\right)}{\p x}} \\[5pt]
  \implies\quad
  &\overline{
    \bar{u} \f{\p \bar{u}}{\p x} + \bar{u} \f{\p u^\prime}{\p x}
    + u^\prime \f{\p \bar{u}}{\p x} + u^\prime \f{\p u^\prime}{\p x}
  } \\[5pt]
  \implies\quad
  &\overline{\bar{u} \f{\p \bar{u}}{\p x}}
  + \overline{\bar{u} \f{\p u^\prime}{\p x}}
  + \overline{u^\prime \f{\p \bar{u}}{\p x}}
  + \overline{u^\prime \f{\p u^\prime}{\p x}}
  \label{eq:NS-time-averaged-rhs-nonlinear-first-term-x}
  \end{aligned}
\end{equation}
%
There are four terms we need to calculate the time-average for:
%
\begin{equation}
  \begin{gathered}
    \overline{\bar{u} \f{\p \bar{u}}{\p x}}
    = \bar{u} \overline{\f{\p \bar{u}}{\p x}}
    = \bar{u} \f{\p \overline{\bar{u}}}{\p x}
    = \bar{u} \f{\p \bar{u}}{\p x} \\[5pt]
    %
    \overline{\bar{u} \f{\p u^\prime}{\p x}}
    = \bar{u} \overline{\f{\p u^\prime}{\p x}}
    = \bar{u} \f{\p \bcancel{\overline{u^\prime}}}{\p x}
    = 0 \\[5pt]
    %
    \overline{u^\prime \f{\p \bar{u}}{\p x}}
    \overset{*}{=} \f{\p \bar{u}}{\p x} \bcancel{\overline{u^\prime}} = 0 \\[5pt]
    %
    \overline{u^\prime \f{\p u^\prime}{\p x}} \overset{**}{\neq} 0
  \end{gathered}
  \label{eq:NS-time-averaged-rhs-nonlinear-first-term-x-expanded}
\end{equation}
%
\textbf{Note (*)}: average velocity gradient, \(\f{\p \bar{u}}{\p x}\), does not change in the period in which the gradient is time-averaged.
Thus the gradient of the average velocity is a constant term inside the integral.
%
\textbf{(**)}: This can't be simplified any further or dropped to zero.
%

Similarly, for the second and third nonlinear terms we will have,
%
\begin{equation}
  \begin{aligned}
    &\overline{\left(\bar{v} + v^\prime\right) \f{\p \left(\bar{u} + u^\prime\right)}{\p y}} \\[5pt]
    \implies\quad
    &\overline{
      \bar{v} \f{\p \bar{u}}{\p y} + \bar{v} \f{\p u^\prime}{\p y}
      + v^\prime \f{\p \bar{u}}{\p y} + v^\prime \f{\p u^\prime}{\p y}
    } \\[5pt]
    \implies\quad
    &\overline{\bar{v} \f{\p \bar{u}}{\p y}}
    + \overline{\bar{v} \f{\p u^\prime}{\p y}}
    + \bcancel{\overline{\bar{v} \f{\p u^\prime}{\p y}}}
    + \bcancel{\overline{v^\prime \f{\p \bar{u}}{\p y}}} \\[5pt]
    \implies\quad
    &\bar{v} \f{\p \bar{u}}{\p y} + \overline{v^\prime \f{\p u^\prime}{\p y}}
  \end{aligned}
  \label{eq:NS-time-averaged-rhs-nonlinear-first-term-v}
\end{equation}
%
\begin{equation}
  \begin{aligned}
    &\overline{\left(\bar{w} + w^\prime\right) \f{\p \left(\bar{u} + u^\prime\right)}{\p z}} \\[5pt]
    \implies\quad
    &\overline{
      \bar{w} \f{\p \bar{u}}{\p z} + \bar{w} \f{\p u^\prime}{\p z}
      + w^\prime \f{\p \bar{u}}{\p z} + w^\prime \f{\p u^\prime}{\p z}
    } \\[5pt]
    \implies\quad
    &\overline{\bar{w} \f{\p \bar{u}}{\p z}}
    + \bcancel{\overline{\bar{w} \f{\p u^\prime}{\p z}}}
    + \bcancel{\overline{w^\prime \f{\p \bar{u}}{\p z}}}
    + \overline{w^\prime \f{\p u^\prime}{\p z}} \\[5pt]
    \implies\quad
    &\bar{w} \f{\p \bar{u}}{\p z} + \overline{w^\prime \f{\p u^\prime}{\p z}}
  \end{aligned}
  \label{eq:NS-time-averaged-rhs-nonlinear-first-term-w}
\end{equation}


Therefore, the right hand side of the x component of the time-averaged Navier-Stokes equation will be:
%
\begin{equation}
  \begin{aligned}
    RHS &= \rho \left(
      \f{\p \bar{u}}{\p t}
      + \bar{u} \f{\p \bar{u}}{\p x} + \overline{u^\prime \f{\p u^\prime}{\p x}}
      + \bar{v} \f{\p \bar{u}}{\p y} + \overline{v^\prime \f{\p u^\prime}{\p y}}
      + \bar{w} \f{\p \bar{u}}{\p z} + \overline{w^\prime \f{\p u^\prime}{\p z}}
    \right) \\[5pt]
      RHS &= \rho \left(
        \f{\p \bar{u}}{\p t}
        + \bar{u} \f{\p \bar{u}}{\p x}
        + \bar{v} \f{\p \bar{u}}{\p y}
        + \bar{w} \f{\p \bar{u}}{\p z}
        + \overline{u^\prime \f{\p u^\prime}{\p x}}
        + \overline{v^\prime \f{\p u^\prime}{\p y}}
        + \overline{w^\prime \f{\p u^\prime}{\p z}}
      \right)
  \end{aligned}
  \label{eq:NS-time-averaged-rhs-combined}
\end{equation}

Considering the following relation, the advection term can be transformed into a new form

\begin{equation}
  \begin{aligned}
       \bar{u} \f{\p \bar{u}}{\p x}
    +  \bar{v} \f{\p \bar{v}}{\p y}
    +  \bar{w} \f{\p \bar{w}}{\p z}
    &= \f{\p \left({\bar{u}}^2\right)}{\p x}
    +  \f{\p \left({\bar{v}}^2\right)}{\p y}
    +  \f{\p \left({\bar{w}}^2\right)}{\p z}
    -  \bar{u} \underbrace{\left( \f{\p \bar{u}}{\p x} + \f{\p \bar{v}}{\p y} + \f{\p \bar{w}}{\p z}\right)}_{=\;0\text{ (Eq.~\ref{eq:continuity-time-averaged-final-avg})}} \\[5pt]
    &= \f{\p \left({\bar{u}}^2\right)}{\p x} + \f{\p \left(\bar{u}\bar{v}\right)}{\p y} + \f{\p \left(\bar{u}\bar{w}\right)}{\p z}
  \end{aligned}
  \label{eq:advection-term-transformation-avg}
\end{equation}
%
And similarly for the fluctuating terms:
\begin{equation}
  \begin{aligned}
       {u^\prime} \f{\p {u^\prime}}{\p x}
    +  {v^\prime} \f{\p {v^\prime}}{\p y}
    +  {w^\prime} \f{\p {w^\prime}}{\p z}
    &= \f{\p \left({{u^\prime}}^2\right)}{\p x}
    +  \f{\p \left({{v^\prime}}^2\right)}{\p y}
    +  \f{\p \left({{w^\prime}}^2\right)}{\p z}
    -  {u^\prime} \underbrace{\left( \f{\p {u^\prime}}{\p x} + \f{\p {v^\prime}}{\p y} + \f{\p {w^\prime}}{\p z}\right)}_{=\;0\text{ (Eq.~\ref{eq:continuity-time-averaged-final-fluctuating})}} \\[5pt]
    &= \f{\p \left({{u^\prime}}^2\right)}{\p x} + \f{\p \left({u^\prime}{v^\prime}\right)}{\p y} + \f{\p \left({u^\prime}{w^\prime}\right)}{\p z}
  \end{aligned}
  \label{eq:advection-term-transformation-fluctuating}
\end{equation}
%
Now, with the transformation in the Eq.~\ref{eq:advection-term-transformation-avg} and Eq.~\ref{eq:advection-term-transformation-fluctuating}, the right hand side of the Navier-Stokes equation, or Eq.~\ref{eq:NS-time-averaged-rhs-combined}, turns into the following equation
%
\begin{equation}
  \begin{aligned}
    RHS &= \rho \left(
    \f{\p \bar{u}}{\p t}
    %
    + \f{\p \left({\bar{u}}^2\right)}{\p x}
    + \f{\p \left(\bar{u}\bar{v}\right)}{\p y}
    + \f{\p \left(\bar{u}\bar{w}\right)}{\p z}
    %
    + \overline{\f{\p \left({{u^\prime}}^2\right)}{\p x}}
    + \overline{\f{\p \left({u^\prime}{v^\prime}\right)}{\p y}}
    + \overline{\f{\p \left({u^\prime}{w^\prime}\right)}{\p z}}
  \right)\\
  &= \rho \left(
    \f{\p \bar{u}}{\p t}
    %
    + \f{\p \left({\bar{u}}^2\right)}{\p x}
    + \f{\p \left(\bar{u}\bar{v}\right)}{\p y}
    + \f{\p \left(\bar{u}\bar{w}\right)}{\p z}
    %
    + \f{\p \left(\overline{{{u^\prime}}^2}\right)}{\p x}
    + \f{\p \left(\overline{{u^\prime}{v^\prime}}\right)}{\p y}
    + \f{\p \left(\overline{{u^\prime}{w^\prime}}\right)}{\p z}
  \right)\\
  \end{aligned}
  \label{eq:NS-time-averaged-rhs-combined-transformed}
\end{equation}
%
This form is more commonly seen in literature.

Now, we combine the relation for the right-hand-side (Eq.~\ref{eq:NS-time-averaged-rhs-combined-transformed}) and left-hand-side (Eq.~\ref{eq:NS-time-averaged-lhs}) of the time-averaged Navier-Stokes equation, to complete the x-momentum equation,

\begin{multline}
  \mu \nabla^2 \left(\bar{u}\right) - \f{\p \left(\bar{p}\right)}{\p x} + \rho g_x \\
  %
  =
  %
  \rho \left(
    \f{\p \bar{u}}{\p t}
    %
    + \f{\p \left({\bar{u}}^2\right)}{\p x}
    + \f{\p \left(\bar{u}\bar{v}\right)}{\p y}
    + \f{\p \left(\bar{u}\bar{w}\right)}{\p z}
    %
    \colorboxed{orange}{
    + \f{\p \left(\overline{{{u^\prime}}^2}\right)}{\p x}
    + \f{\p \left(\overline{{u^\prime}{v^\prime}}\right)}{\p y}
    + \f{\p \left(\overline{{u^\prime}{w^\prime}}\right)}{\p z}
    }
  \right).
  \label{eq:NS-time-averaged-x}
\end{multline}
%
For the \(y\) and \(z\) direction we also have:
%
\begin{multline}
  \mu \nabla^2 \left(\bar{v}\right) - \f{\p \left(\bar{p}\right)}{\p y} + \rho g_y \\
  %
  =
  %
  \rho \left(
    \f{\p \bar{v}}{\p t}
    %
    + \f{\p \left(\bar{u}\bar{v}\right)}{\p x}
    + \f{\p \left({\bar{v}}^2\right)}{\p y}
    + \f{\p \left(\bar{v}\bar{w}\right)}{\p z}
    %
    \colorboxed{orange}{
      + \f{\p \left(\overline{{u^\prime}{v^\prime}}\right)}{\p x}
      + \f{\p \left(\overline{{{v^\prime}}^2}\right)}{\p y}
      + \f{\p \left(\overline{{v^\prime}{w^\prime}}\right)}{\p z}
    }
  \right).
  \label{eq:NS-time-averaged-y}
\end{multline}
%
\begin{multline}
  \mu \nabla^2 \left(\bar{w}\right) - \f{\p \left(\bar{p}\right)}{\p z} + \rho g_z \\
  %
  =
  %
  \rho \left(
    \f{\p \bar{w}}{\p t}
    %
    + \f{\p \left(\bar{u}\bar{w}\right)}{\p x}
    + \f{\p \left(\bar{v}\bar{w}\right)}{\p y}
    + \f{\p \left({\bar{w}}^2\right)}{\p z}
    %
    \colorboxed{orange}{
      + \f{\p \left(\overline{{u^\prime}{w^\prime}}\right)}{\p x}
      + \f{\p \left(\overline{{v^\prime}{w^\prime}}\right)}{\p y}
      + \f{\p \left(\overline{{{w^\prime}}^2}\right)}{\p z}
    }
  \right).
  \label{eq:NS-time-averaged-z}
\end{multline}

Note that all terms in the time-averaged equation (including partial derivatives and such) are all constant over the period in which the time-integral is being performed.
%
The three time-averaged fluctuating term on the right-hand-side of the Eq.~\ref{eq:NS-time-averaged-x} contain terms that are called ``Reynolds' stress terms''. These stress term, as their name suggest, have the unit of stress: \([N/m^2]\),
%
\begin{equation}
  \begin{aligned}
    \tau^\prime_{ij}
    &= \rho \overline{u^\prime_i u^\prime_j} \\[5pt]
    &= \begin{pmatrix}
      \overline{{{u^\prime}}^2}
    & \overline{{u^\prime}{v^\prime}}
    & \overline{{u^\prime}{w^\prime}} \\
    %
      \overline{{u^\prime}{v^\prime}}
    & \overline{{{v^\prime}}^2}
    & \overline{{v^\prime}{w^\prime}} \\
    %
      \overline{{u^\prime}{w^\prime}}
    & \overline{{v^\prime}{w^\prime}}
    & \overline{{{w^\prime}}^2}
    \end{pmatrix}
  \end{aligned}
  \label{eq:reynolds-stress-matrix}
\end{equation}
%
\begin{quote}
  \emph{``In fluid dynamics, the Reynolds stress is the component of the total stress tensor in a fluid obtained from the averaging operation over the Navier–Stokes equations to account for turbulent fluctuations in fluid momentum.''}--Wikipedia
\end{quote}
%
The time-averaged equations can be solved if the Reynolds stresses can be related to the mean flow quantities.
This is known as the ``\textbf{closure problem}''.



\section{Boussinesq Assumption}\label{sec:boussinesq-assumption}

The \emph{Boussinesq assumption} (aka \emph{Boussinesq approximation}, \emph{Boussinesq hypothesis} and sometimes \emph{eddy viscosity model}) was proposed by Boussinesq in 1877. He postulated that the Reynolds stresses could be linked to the mean rates of deformation.
%
\begin{equation}
  \begin{gathered}
    -\rho \overline{u^\prime u^\prime} = 2 \mu_T \f{\p \bar{u}}{\p x} - \f{2}{3} \rho k,\quad
    -\rho \overline{u^\prime v^\prime} = \mu_T \left( \f{\p \bar{v}}{\p x} + \f{\p \bar{u}}{\p y} \right),\quad
    -\rho \overline{u^\prime w^\prime} = \mu_T \left( \f{\p \bar{w}}{\p x} + \f{\p \bar{u}}{\p z} \right), \\
    %
    -\rho \overline{u^\prime v^\prime} = \mu_T \left( \f{\p \bar{v}}{\p x} + \f{\p \bar{u}}{\p y} \right),\quad
    -\rho \overline{v^\prime v^\prime} = 2 \mu_T \f{\p \bar{v}}{\p y} - \f{2}{3} \rho k,\quad
    -\rho \overline{v^\prime w^\prime} = \mu_T \left( \f{\p \bar{w}}{\p y} + \f{\p \bar{v}}{\p z} \right), \\
    %
    -\rho \overline{u^\prime w^\prime} = \mu_T \left( \f{\p \bar{u}}{\p z} + \f{\p \bar{w}}{\p z} \right),\quad
    -\rho \overline{v^\prime w^\prime} = \mu_T \left( \f{\p \bar{v}}{\p z} + \f{\p \bar{w}}{\p y} \right),\quad
    -\rho \overline{w^\prime w^\prime} = 2 \mu_T \f{\p \bar{w}}{\p z} - \f{2}{3} \rho k,
  \end{gathered}
  \label{eq:Boussinesq-approximation}
\end{equation}
%
The right-hand side is analogous to \emph{Newton’s law of viscosity} (\(\tau = \mu\,{\p u}/{\p y}\)), except for the appearance of the \textbf{turbulent} or \textbf{eddy viscosity}, \(\mu_T\), and \textbf{turbulent kinetic energy}, \(k\).
Note that \(\mu_T\) is artificial and controls the \textbf{strength} of the diffusion.
This means that when the turbulence is stronger, there's a greater transform of momentum from faster moving layers to slower moving layers.
This diffusion is always in the same direction as the mean velocity gradient.
Note, also, that Further modeling is needed to calculate \(\mu_T\).
\(k\) is obtained via the following equation:
%
\begin{equation}
  \colorboxed{orange}{
    k = \f{1}{2} \left(\overline{u^\prime u^\prime} + \overline{v^\prime v^\prime} + \overline{w^\prime w^\prime}\right)
  }
  \label{eq:turbulent-kinetic-energy}
\end{equation}
%
We can simplify all the above relations in the Eq.~\ref{eq:Boussinesq-approximation}
%
\begin{equation}
  \colorboxed{orange}{
    -\overline{u^\prime_i u^\prime_j} = \nu_T \left( \f{\p \bar{u_i}}{\p x_j} + \f{\bar{u_j}}{\p x_i}\right) - \f{2}{3} k \delta_{ij}
  }
  \label{eq:Boussinesq-approximation-index-notation}
\end{equation}
%
where \(\delta_{ij}\) is the Kronecker delta, which is equal to \(1\) when \(i = j\) and \(0\) otherwise.

The value for \(\mu_T\) and \(k\) is obtained based on experiments and through curve fitting the best value is chosen.
Many turbulence models build on top of this assumption, e.g., the \keModel model.
From here on out, we assume the flow is isotropic and thus \(\mu_T\), and \(k\) are constant and direction independent.

\subsection{Small note on normal stresses}

Unlike shear Reynolds stresses, the normal stresses have an extra term, \(\f{2}{3} \rho k\).
In this subsection we'll explain why this extra term exists
[notes in this section is taken from \href{https://youtu.be/SVYXNICeNWA}{\color{blue}this} video].

If we assume, just similar to shear stresses that normal Reynolds stress is as follows,
%
\begin{equation}
  \rho \overline{u^\prime u^\prime} = 2 \mu_T \f{\bar{u}}{\p x},
  \label{eq:normal-Reynolds-stree-incorrect}
\end{equation}
%
Then we will have a small problem...
The definition of the turbulent kinetic energy tells us
%
\begin{equation}
    k = \f{1}{2} \left(\overline{u^\prime u^\prime} + \overline{v^\prime v^\prime} + \overline{w^\prime w^\prime}\right).
\end{equation}
%
Hence, the sum of the normal turbulent stresses should give
%
\begin{equation}
  \begin{aligned}
    -\left(\rho \overline{u^\prime u^\prime} + \rho \overline{v^\prime v^\prime} + \rho \overline{w^\prime w^\prime}\right)
    &= -\rho \left(\overline{u^\prime u^\prime} + \overline{v^\prime v^\prime} + \overline{w^\prime w^\prime}\right) \\
    &= -2 \rho k.
  \end{aligned}
\end{equation}
%
However, if we add our three normal stresses together we get:
%
\begin{equation}
   -\rho \left(\overline{u^\prime u^\prime} + \overline{v^\prime v^\prime} + \overline{w^\prime w^\prime}\right) \\
   =
   2 \mu_T \left(\f{\p \bar{u}}{\p x} + \f{\p \bar{v}}{\p y} + \f{\p \bar{w}}{\p z} \right).
\end{equation}
%
If the flow is incompressible then the continuity equation tells us that,
%
\begin{equation}
  \f{\p \bar{u}}{\p x} + \f{\p \bar{v}}{\p y} + \f{\p \bar{w}}{\p z} = 0.
\end{equation}
%
Hence:
%
\begin{equation}
   -\rho \left(\overline{u^\prime u^\prime} + \overline{v^\prime v^\prime} + \overline{w^\prime w^\prime}\right)
   = 0\quad \text{[Incompressible Flows]}.
\end{equation}
%
This is clearly incorrect!
To correct the calculation of the normal stresses, we subtract 1/3rd of the error from each of the normal components.
%
\begin{equation}
  \begin{aligned}
  -\rho \overline{u^\prime u^\prime}
  &=
  2 \mu_T \left(
            \f{\p \bar{u}}{\p x}
            - \f{1}{3} \left(\f{\p \bar{u}}{\p x} + \f{\p \bar{v}}{\p y} + \f{\p \bar{w}}{\p z}\right)
          \right)
  - \f{1}{3}\left(2 \rho k\right) \\
  %
  -\rho \overline{v^\prime v^\prime}
  &=
  2 \mu_T \left(
            \f{\p \bar{v}}{\p y}
            - \f{1}{3} \left(\f{\p \bar{u}}{\p x} + \f{\p \bar{v}}{\p y} + \f{\p \bar{w}}{\p z}\right)
          \right)
  - \f{1}{3}\left(2 \rho k\right) \\
  %
  -\rho \overline{w^\prime w^\prime}
  &=
  2 \mu_T \left(
            \f{\p \bar{w}}{\p z}
            - \f{1}{3} \left(\f{\p \bar{u}}{\p x} + \f{\p \bar{v}}{\p y} + \f{\p \bar{w}}{\p z}\right)
          \right)
  - \f{1}{3}\left(2 \rho k\right)
  \end{aligned}
\end{equation}
%
As a check, we add the normal stresses together and we get the turbulent kinetic energy!
%
\begin{equation}
   -\rho \left(\overline{u^\prime u^\prime} + \overline{v^\prime v^\prime} + \overline{w^\prime w^\prime}\right)
   =
   2 \mu_T
   \left(
     \bcancel{\f{\p \bar{u}}{\p x} + \f{\p \bar{v}}{\p y} + \f{\p \bar{w}}{\p z}}
     - \bcancel{\left(\f{\p \bar{u}}{\p x} + \f{\p \bar{v}}{\p y} + \f{\p \bar{w}}{\p z}\right)}
   \right)
   - 2 \rho k,
\end{equation}
%
\begin{equation}
   -\rho \left(\overline{u^\prime u^\prime} + \overline{v^\prime v^\prime} + \overline{w^\prime w^\prime}\right)
   =
   - 2 \rho k,
\end{equation}
%
\begin{equation}
  \colorboxed{gray}{
    k = \f{1}{2} \left(\overline{u^\prime u^\prime} + \overline{v^\prime v^\prime} + \overline{w^\prime w^\prime}\right).
  }
\end{equation}


\subsection{Compressible flow}

All the equations discussed in Section~\ref{sec:boussinesq-assumption} were derived for the incompressible flow.
For compressible flow, the velocity field is not divergence free and thus we have a different equation for normal stress terms.
%
\begin{equation}
  \begin{aligned}
    \text{shear stress:}&\quad
    -\rho \overline{u^\prime_i u^\prime_j} = \mu_T \left(\f{\p \bar{u}_i}{\p x_j} + \f{\p \bar{u}_j}{\p x_i}\right), \\
  %
  \text{normal stress:}&\quad
    -\rho \overline{u^\prime_i u^\prime_j}
    =
    \mu_T \left(\f{\p \bar{u}_i}{\p x_j} + \f{\p \bar{u}_j}{\p x_i} - \f{2}{3} \f{\p \bar{u}_k}{\p x_k} \right)
    - \f{2}{3} \rho k; \quad (i = j).
  \end{aligned}
  \label{eq:reynolds-stress-compressible-flow}
\end{equation}
%
Note that the repeated indices (\(k\),\(k\)) indicate summation/contraction.
We can take this a step further and express the Reynolds stress in one equation, by use of the Kronecker delta, \(\delta_{ij}\).
We multiply the extra terms in the normal components by the Kronecker delta.
They will vanish when \(i \neq j\)!
Hence we can bring everything together:
%
\begin{equation}
  \colorboxed{orange}{
    -\rho \overline{u^\prime_i u^\prime_j}
    =
    \mu_T \left(\f{\p \bar{u}_i}{\p x_j} + \f{\p \bar{u}_j}{\p x_i} - \f{2}{3} \f{\p \bar{u}_k}{\p x_k} \delta_{ij} \right)
    - \f{2}{3} \rho k \delta_{ij}.
  }
  \label{eq:reynolds-stress-term-compressible-flow-combined}
\end{equation}


\subsection{Vector form}

The vector form of the Eq.~\ref{eq:reynolds-stress-term-compressible-flow-combined} is the following:
%
\begin{equation}
  -\rho \overline{\mb{U}^\prime \mb{U}^\prime}
  =
  \mu_T \left(
    (\nabla \mb{U}) + (\nabla \mb{U})^T - \f{2}{3} (\nabla \cdot \mb{U}) \mb{I}
  \right)
  - \f{2}{3} \rho k \mb{I}
  \label{eq:reynolds-stress-term-vector-form}
\end{equation}


\subsection{Strain Rate Tensor}

We can use the mean rate of strain tensor, \(S_{ij}\), which is defined as:
%
\begin{equation}
  S_{ij} = \f{1}{2} \left( \f{\p \bar{u}_i}{\p x_j} + \f{\p \bar{u}_j}{\p x_i} \right),
  \label{eq:mean-strain-rate-tensor}
\end{equation}
%
or even more compact, we can use the \emph{deviatoric part} (or trace-less strain rate tensor) of this tensor \(S^*_{ij}\):
%
\begin{equation}
  S^*_{ij}
  = \f{1}{2}
  \left(
    \f{\p \bar{u}_i}{\p x_j}
    + \f{\p \bar{u}_j}{\p x_i}
    - \f{1}{3} \f{\p \bar{u}_k}{\p x_k} \delta_{ij}
  \right).
  \label{eq:mean-strain-rate-tensor-deviatoric-part}
\end{equation}
%
Many authors use the deviatoric part to make the equation compact.
It has no special meaning.
We will have the following two equations for the Reynolds stress terms based on the Eq.~\ref{eq:mean-strain-rate-tensor} and Eq.~\ref{eq:mean-strain-rate-tensor-deviatoric-part}, respectively
%
\begin{equation}
  \colorboxed{orange}{
    -\rho \overline{u^\prime_i u^\prime_j}
    =
    2 \mu_T \left(S_{ij} - \f{1}{3} \f{\p \bar{u}_k}{\p x_k} \delta_{ij} \right)
    - \f{2}{3} \rho k \delta_{ij}.
  }
  \label{eq:reynolds-stress-term-with-strain-rate-tensor}
\end{equation}
%
\begin{equation}
  \colorboxed{orange}{
    -\rho \overline{u^\prime_i u^\prime_j}
    =
    2 \mu_T S^*_{ij}
    - \f{2}{3} \rho k \delta_{ij}.
  }
  \label{eq:reynolds-stress-term-with-strain-rate-tensor-deviatoric-part}
\end{equation}
%
You will see both of these definitions in the literature.


\subsection{OpenFOAM}

In OpenFOAM you can find the formula in \mono{eddyViscosity.C}:
%
\begin{minted}[escapeinside=||]{cpp}
((2.0/3.0)*|\sfhlmono{I}|)*tk() - (nut_)*|\sfhlmono{dev}|(twoSymm(fvc::grad(this->U_)))
\end{minted}
%
\mono{dev} indicates the deviatoric part, and \mono{I} is the identity matrix / Kronecker delta.


\begin{listing}[H]
  \renewcommand\theFancyVerbLine{%
    \ifnum\value{FancyVerbLine}=86
      \boxed{\texttt{\arabic{FancyVerbLine}}}
    \else
      \ifnum\value{FancyVerbLine}=120
        \boxed{\texttt{\arabic{FancyVerbLine}}}
      \else
        \textcolor{black}{\color{MidGrey}\arabic{FancyVerbLine}}%
      \fi
    \fi
  }
\begin{minted}[linenos,firstnumber=84,highlightlines={120},escapeinside=||]{cpp}
template<class BasicTurbulenceModel>
Foam::tmp<Foam::volSymmTensorField>
Foam::eddyViscosity<BasicTurbulenceModel>::|\sfhlmono{R()}| const
{
  tmp<volScalarField> |\sfhlmono{tk(k())}|;

    // Get list of patchField type names from k
    wordList patchFieldTypes(tk().boundaryField().types());

    // For k patchField types which do not have an equivalent for symmTensor
    // set to calculated
    forAll(patchFieldTypes, i)
    {
        if
        (
           !fvPatchField<symmTensor>::patchConstructorTablePtr_
                ->found(patchFieldTypes[i])
        )
        {
            patchFieldTypes[i] = fvPatchFieldBase::calculatedType();
        }
    }

    return tmp<volSymmTensorField>
    (
        new |\sfhlmono{volSymmTensorField}|
        (
            IOobject
            (
                IOobject::groupName("R", this->alphaRhoPhi_.group()),
                this->runTime_.timeName(),
                this->mesh_,
                IOobject::NO_READ,
                IOobject::NO_WRITE,
                IOobject::NO_REGISTER
            ),
            ((2.0/3.0)*I)*tk() - (nut_)*devTwoSymm(fvc::grad(this->U_)),
            patchFieldTypes
        )
    );
}
\end{minted}
\caption{\small{\protect\path{$FOAM_SRC/TurbulenceModels/turbulenceModels/eddyViscosity/eddyViscosity.C}}}
\end{listing}
%
Also, In the \mono{eddyViscosity.H}, we have the following:
%
% |\setcounter{FancyVerbLine}{1}|
\begin{listing}[H]
  \renewcommand\theFancyVerbLine{%
    \ifnum\value{FancyVerbLine}=103
      \setcounter{FancyVerbLine}{107}\vdots
    \else
      \ifnum\value{FancyVerbLine}=113
        \setcounter{FancyVerbLine}{118}\vdots
      \else
        \ifnum\value{FancyVerbLine}=124
          \vdots
        \else
          \ifnum\value{FancyVerbLine}=101
            \vdots
          \else
            \color{MidGrey}\arabic{FancyVerbLine}%
          \fi
        \fi
      \fi
    \fi
  }
\begin{minted}[linenos,firstnumber=101]{cpp}

    // Member Functions

        //- Return the turbulence viscosity
        virtual tmp<volScalarField> nut() const
        {
          return nut_;
        }

        //- Return the turbulence kinetic energy
        virtual tmp<volScalarField> k() const = 0;

        //- Return the Reynolds stress tensor
        virtual tmp<volSymmTensorField> R() const;

\end{minted}
\caption{{\protect\path{$FOAM_SRC/TurbulenceModels/turbulenceModels/eddyViscosity/eddyViscosity.H}}}
\end{listing}



% \textorpdfstring{math} to fix the issue of hyperref throwing warning when math is in the section title
\subsection{How \texorpdfstring{\(\mu_T\)}\ \ is calculated?}

\(mu_T\) is calculated with the chosen turbulence model (like \keModel).
We solve an eqaution for \(k\) and \(\epsilon\), which allows us to calculate \(mu_T\)
%
\begin{equation}
  \mu_T = \rho C_{\mu} \f{k^2}{\epsilon}
  \qquad
  [k\text{-}\epsilon]
  \label{eq:mu_t-formula-k-epsilon}
\end{equation}
%
Once \(\mu_T\) has been calculated, use the eddy viscosity hypothesis to calculate the Reynolds stresses.
%
\begin{equation}
  -\rho \overline{u^\prime_i u^\prime_j} = 2 \mu_T S^*_{ij} - \f{2}{3} \rho k \delta_{ij}
\end{equation}
%
Now the RANS equations can be solved.


\subsection{Limitations}

Craft et al. \cite{craft1996development} show some limitations of the linear eddy viscosity models (like \keModel)


\subsection{Iostropic Turbulence}

If the turbulent fluctuations are completely isotropic, that is, if they do not have any directional preference, then the off-diagonal components of u i u j ¯ vanish, and the normal stresses are equal.
(\url{https://www.sciencedirect.com/topics/engineering/isotropic-turbulence})
