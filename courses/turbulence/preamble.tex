\usepackage[a4paper,margin=1in]{geometry}

\usepackage[utf8]{inputenc}
% \usepackage{times}
% \usepackage{mathptmx}
\usepackage[T1]{fontenc}

\usepackage[english]{babel}

\usepackage[square,numbers]{natbib}
\bibliographystyle{unsrtnat}

\usepackage[hidelinks]{hyperref}
\pdfstringdefDisableCommands{\def\varepsilon{\textepsilon}}
\usepackage{bookmark}% faster updated bookmarks


% Math stuff
\usepackage{amsmath, amsfonts, mathtools, amsthm, amssymb}
\usepackage{nicefrac}

\newcommand{\f}{\frac}
\newcommand{\nf}{\nicefrac}
\newcommand{\mb}{\mathbf}
\newcommand{\degree}{\ensuremath{^\circ\,}}
\newcommand{\tr}[1]{\mathrm{Tr}\left({#1}\right)}
\newcommand{\mbg}[1]{\boldsymbol{\mathbf{#1}}}
\renewcommand{\d}{\mathop{}\!\mathrm{d}} % total derivative
\newcommand{\D}{\mathop{}\!\mathrm{D}} % total derivative
\newcommand{\p}{\partial}

% Put x \to \infty below \lim
\let\svlim\lim\def\lim{\svlim\limits}
%Make implies and impliedby shorter
\let\implies\Rightarrow
\let\impliedby\Leftarrow
\let\iff\Leftrightarrow

% Swap epsilon and varepsilon
\let\epsilonOld\epsilon
\let\epsilon\varepsilon   %<------- epsilon == varepsilon
\let\varepsilon\epsilonOld

% Add \contra symbol to denote contradiction
\usepackage{stmaryrd} % for \lightning
\newcommand\contra{\scalebox{1.5}{$\lightning$}}


% Fancy script capitals
\usepackage{mathrsfs}
\usepackage{cancel}
% Bold math
\usepackage{bm}
% Some shortcuts
\newcommand\N{\ensuremath{\mathbb{N}}}
\newcommand\R{\ensuremath{\mathbb{R}}}
\newcommand\Z{\ensuremath{\mathbb{Z}}}
\renewcommand\O{\ensuremath{\emptyset}}
\newcommand\Q{\ensuremath{\mathbb{Q}}}
\newcommand\C{\ensuremath{\mathbb{C}}}



% Figures
\usepackage{graphicx}
\graphicspath{{./figures}}

% Figure support as explained in my blog post.
\usepackage{import}
\usepackage{xifthen}
\usepackage{pdfpages}
\usepackage{transparent}
\newcommand{\incfig}[1]{%
    \def\svgwidth{\columnwidth}
    \import{./figures/}{#1.pdf_tex}
}


% Code listings
\usepackage{listings}
\definecolor{LightGrey}{HTML}{fafafa}
% \lstset{ %
%     escapeinside={(*@}{@*)},         % escape characters to write latex code in the listing (e.g. highlighting) (  (*@  \textcolor{blue}{code}  @*) )
%     breaklines=true,                 % automatic line breaking only at whitespace
%     captionpos=b,                    % sets the caption-position to bottom
%     backgroundcolor=\color{LightGrey},
%     basicstyle=\footnotesize\ttfamily,
%     % numbers=left,
%     % stepnumber=1,
% }
\definecolor{codegreen}{rgb}{0,0.6,0}
\definecolor{codegray}{rgb}{0.5,0.5,0.5}
\definecolor{codepurple}{rgb}{0.58,0,0.82}
\definecolor{backcolour}{rgb}{0.95,0.95,0.92}
\lstdefinestyle{mystyle}{
    backgroundcolor=\color{backcolour},
    commentstyle=\color{codegreen},
    keywordstyle=\color{magenta},
    numberstyle=\tiny\color{codegray},
    stringstyle=\color{codepurple},
    basicstyle=\ttfamily\footnotesize,
    breakatwhitespace=false,
    breaklines=true,
    captionpos=b,
    keepspaces=true,
    numbers=left,
    numbersep=5pt,
    showspaces=false,
    showstringspaces=false,
    showtabs=false,
    tabsize=2,
    escapeinside={/@}{@/)}, % escape characters to write latex code in the listing (e.g. highlighting) (  (*@  \textcolor{blue}{code}  @*) )
    % columns=fullflexible,  % fixes issues related to spacing of the tex code inside lstlistings
    moredelim=[is][\underbar]{_}{_},
}
\lstset{ style=mystyle }

\usepackage{color} % for highlight purpose
\usepackage[cache=false,outputdir=build]{minted}
\definecolor{LightBrown}{rgb}{0.95,0.95,0.92}
\definecolor{LightGrey}{HTML}{fafafa}
\definecolor{MidGrey}{HTML}{8a8a8a}
% \setminted[<language>]{<key=value,...>}
\setminted{
fontsize=\footnotesize,
bgcolor=LightBrown,
}
\renewcommand{\theFancyVerbLine}{\textcolor{MidGrey}{\arabic{FancyVerbLine}}}
% reduce the space between caption and listing
\AtEndEnvironment{listing}{\vspace{-20pt}}

% CUSTOM COMMANDS
\newcommand{\mono}{\texttt}


% Other

% Units
\usepackage{siunitx}

% Paths and urls
% xurl: a seemingly better version of url, which solves the problem long urls/paths not breaking.
\usepackage{xurl}


% Colored boxed for amsmath
%
% https://tex.stackexchange.com/a/326380/220469
%
% Definition of \boxed in amsmath.sty:
% \newcommand{\boxed}[1]{\fbox{\m@th$\displaystyle#1$}}
\usepackage{xcolor}
\definecolor{LightBlue}{HTML}{e6feff}
%stroke-fill-hl
\newcommand{\sfhl}{\fcolorbox{LightBlue}{LightBlue}}
%stroke-fill-hl-mono (mono version)
\newcommand{\sfhlmono}[1]{\fcolorbox{LightBlue}{LightBlue}{\texttt{#1}}}

% Syntax: \colorboxed[<color model>]{<color specification>}{<math formula>}
\newcommand*{\colorboxed}{}
\def\colorboxed#1#{%
  \colorboxedAux{#1}%
}
\newcommand*{\colorboxedAux}[3]{%
  % #1: optional argument for color model
  % #2: color specification
  % #3: formula
  \begingroup
    \colorlet{cb@saved}{.}%
    \color#1{#2}%
    \boxed{%
      \color{cb@saved}%
      #3%
    }%
  \endgroup
}


% Some nice helpers
\usepackage{xspace}
\newcommand{\keModel}{\(k\text{-}\epsilon\)\xspace}

%
% \newcommand{\setbackgroundcolour}{\pagecolor[rgb]{0.19,0.19,0.19}}
% \newcommand{\settextcolour}{\color[rgb]{0.77,0.77,0.77}}
% \newcommand{\invertbackgroundtext}{\setbackgroundcolour\settextcolour}
%
% %Command execution.
% %If this line is commented, then the appearance remains as usual.
% \invertbackgroundtext
